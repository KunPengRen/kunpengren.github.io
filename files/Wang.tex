%%%%%%%%%%%%%%%%%%%%%%%%%%%%%%%%%%%%%%%%%
% Medium Length Professional CV
% LaTeX Template
% Version 2.0 (8/5/13)
%
% This template has been downloaded from:
% http://www.LaTeXTemplates.com
%
% Original author:
% Trey Hunner (http://www.treyhunner.com/)
%
% Important note:
% This template requires the resume.cls file to be in the same directory as the
% .tex file. The resume.cls file provides the resume style used for structuring the
% document.
%
%%%%%%%%%%%%%%%%%%%%%%%%%%%%%%%%%%%%%%%%%

%----------------------------------------------------------------------------------------
%	PACKAGES AND OTHER DOCUMENT CONFIGURATIONS
%----------------------------------------------------------------------------------------

\documentclass{resume} % Use the custom resume.cls style

\usepackage[left=0.75in,top=0.6in,right=0.75in,bottom=0.6in]{geometry} % Document margins
\usepackage[colorlinks,
            linkcolor=red,
            anchorcolor=blue,
            citecolor=green
            ]{hyperref}
\usepackage{verbatim}
\usepackage{enumitem} 
\usepackage{xcolor}
\hypersetup{%
unicode=true, CJKbookmarks=true, bookmarksnumbered=true,
bookmarksopen=true, bookmarksopenlevel=1, breaklinks=true,
colorlinks=false, plainpages=false, pdfpagelabels, pdfborder=0 0 0 }
\name{Xuejian Wang} % Your name
\address{(+1) 412-251-1130 \\ \href{http://xuejianwang.com}{xuejianwang.com} \\ \href{mailto:xuejianw@andrew.cmu.edu}{xuejianw@andrew.cmu.edu}}

\begin{document}
%\begin{CJK*}{GBK}{song}
\vspace{-1em}
%----------------------------------------------------------------------------------------
%	EDUCATION SECTION
%----------------------------------------------------------------------------------------
\begin{rSection}{RESEARCH INTERESTS}
The broad areas of machine learning including deep learning, reinforcement learning and applications in various domains such as recommender systems and network embeddings. In general:
\begin{itemize}[itemindent=-5mm] 
	\itemsep=-5pt
	\vspace*{-5pt}
	\item[・] Mining underlying patterns of human interaction and behaviors
	\item[・] Designing efficient and effective learning algorithms to solve real-world problems
\end{itemize}
\end{rSection}


\begin{rSection}{Education}
%{\bf }%\hfill { } \\
\textbf{Carnegie Mellon University} \hfill \textbf{Pittsburgh, USA}
\vspace{-5pt}
\item[・]Joint PhD in Machine Learning and Public Policy \hfill Sep. 2018 - Present \\ \\
\textbf{Shanghai Jiao Tong University (SJTU)} \hfill \textbf{Shanghai, China}\\ % \smallskip 
B.S in Information Security \hfill Sep. 2014 - Jun. 2018
%\vspace{-5pt} 
%\item[・] GPA: 81.7(Freshman), 83.9(Sophomore), 85.2(Junior)
\vspace{-5pt}
\item[・] Research Assistant, APEX Data \& Knowledge Management Lab
\vspace{-5pt}
\item[・] Advisor: Prof. \href{http://wnzhang.net}{Weinan Zhang}, Prof. Yong Yu and Prof. Jun Wang(University College London)
\end{rSection}

%----------------------------------------------------------------------------------------
%	PUBLICATION SECTION
%----------------------------------------------------------------------------------------
\begin{rSection}{Publications}
%{\bf }%\hfill { } \\
\begin{rSubsection}{Large-scale Interactive Recommendation with Tree-structured Policy Gradient}{}{}{}
\item Haokun Chen, Xinyi Dai, Weinan Zhang, Han Cai, \textbf{Xuejian Wang}, Ruiming Tang, Yuzhou Zhang, Yong Yu
\item In \emph{Proceedings of the 33rd AAAI Conference on Artificial Intelligence (AAAI-19)}. \textbf{AAAI, 2019}
\end{rSubsection}
\vspace{-2pt}
\begin{rSubsection}{Neural link prediction over aligned networks}{}{}{}
\item Xuezhi Cao, Haokun Chen, \textbf{Xuejian Wang}, Weinan Zhang, and Yong Yu.
\item In \emph{Proceedings of the 32nd AAAI Conference on Artificial Intelligence (AAAI-18)}. \textbf{AAAI, 2018}
\end{rSubsection}
\vspace{-2pt}
\begin{rSubsection}{Dynamic attention deep model for article recommendation by learning human editors' demonstration}{}{}{}
\item \textbf{Xuejian Wang}*, Lantao Yu*(equal contribution), Kan Ren, Guanyu Tao, Weinan Zhang, Yong Yu, Jun Wang.
\item In \emph{Proceedings of the 23rd ACM SIGKDD International Conference on Knowledge Discovery and Data Mining}. \textbf{KDD 2017}
\end{rSubsection}
\end{rSection}
%----------------------------------------------------------------------------------------
%	RESEARCH EXPERIENCE SECTION
%----------------------------------------------------------------------------------------
%\vspace{0.1em}

\begin{rSection}{Research Experiences}
\begin{rSubsection}{Personalized Article Recommendation}{Sep. 2017 - Dec. 2017}{Advisor: Prof. Weinan Zhang}{APEX Data $\&$ Knowledge Management Lab, SJTU}
\item This study focuses on recommending most relevant textual advertisements to users to maximize the Click-Through-Rate, in which the major challenge is how to match user interests with proper articles
\item Employing hierarchical attention mechanism to obtain article embeddings and construct user interests
\end{rSubsection}
\begin{rSubsection}{Neural Link Prediction over Aligned Networks}{Aug. 2017 - Sep. 2017}{Advisor: Prof. Yong Yu}{APEX Data $\&$ Knowledge Management Lab, SJTU}
\item Implemented baseline \emph{LINE: Large-scale Information Network Embedding} in Tensorflow for comparison and tuned parameters to best performance
\item Revised the whole paper for 3 times and contributed over 100 submits 
\item Surveyed papers about social networks and extended the idea of two aligned networks to multi-aligned networks which we left as future work
\end{rSubsection}
\vspace{110pt}
\begin{rSubsection}{Dynamic Attention Deep Model for Article Recommendation
by Learning Human Editors’ Demonstration}{Oct. 2016 - Feb. 2017}{Advisor: Prof. Weinan Zhang}{APEX Data $\&$ Knowledge Management Lab, SJTU}
\item Built a text classification network to model the editors' underlying criterion varied with many factors such as time, current affairs, etc., for a famous Chinese media website
\item Employed attention mechanism to address data drift problem, resulting in more robust and stable predictions
\item Proposed a Dynamic Attention Deep Model (DADM) which outperformed other baselines in an A/B test
\item Our paper was accepted to KDD 2017 and the proposed DADM model was utilized in practical cases, automating the quality article selection process to alleviate the editors' working load\\\\
\end{rSubsection}
\end{rSection}
%----------------------------------------------------------------------------------------
%	INTERNSHIP
%----------------------------------------------------------------------------------------
\begin{rSection}{Professional Activities}
\begin{rSubsection}{Review Activities}{CIKM 2017, WSDM 2018, AAAI 2018, SIGIR 2018}{}{}
\item Gave reviews for paper submissions in relevant domains for reference, which were mostly adopted
\item Learnt how to evaluate academic papers from a reviewer perspective and developed tastes for papers 
\end{rSubsection}
\end{rSection}


\begin{rSection}{Internship Experience}
\begin{rSubsection}{ULU Technologies Inc.}{Nov. 2016 - Feb. 2017}{R$\&$D  Engineer Intern}{}
\item Developed a practical algorithm for article recommendation which is used in production 
\item Improved coding ability, learned how to independently conduct experiments and developed communication skills
%\item Developed test cases and wrote general design documents for an industrial reliability testing platform.
%\item Programmed Oracle database in Java for a new testing module to update data and improve efficiency.
\end{rSubsection}
\end{rSection}


%----------------------------------------------------------------------------------------
%	TECHNICAL STRENGTHS SECTION
%----------------------------------------------------------------------------------------

\begin{rSection}{Honors $\&$ Awards}
\begin{rSubsection}{}{}{}{}
\item[] \textbf{KDD Student Travel Award} \hfill{2017}
\item A glorious recognition to KDD attendees awarded by ACM KDD
\item[] \textbf{Rongchang Science and Technology Innovation Scholarship (Nomination)} \hfill{2017}
\item Top 30 undergraduate scholars in Shanghai Jiao Tong University
\item[] \textbf{Shanghai Jiao Tong University Excellent Scholarship} \hfill{2017$\&$2016}
\item Awarded to students for their excellent academic performance
\item[] \textbf{Shanghai Jiao Tong University Excellent Student Award} (Top 5\%)  \hfill{2017$\&$2016}
\item Awarded to top 5\% students for comprehensive performance, selected by peers
\item[] \textbf{Second Prize, China Undergraduate Mathematical Contest in Modeling 2016, Shanghai} \hfill{2016}
\item[] \textbf{Second Prize, ``Data Bang'' Data Innovation Contest (Top 2)}\hfill{2015} 
\end{rSubsection}
\end{rSection}


\begin{rSection}{Skills}
{\bf Machine Learning: }
\hspace*{3.0 cm} Tensorflow, Scikit-Learn, XGBoost, Keras\\
{\bf Programming Languages: }
\hspace*{1.8 cm} Python, MATLAB, C++, R, Verilog and \LaTeX  \\
{\bf Standard Tests: }
\hspace*{3.53 cm} TOEFL: 104 (R26, L27, S22, W29) , GRE: V153, Q170, W3.5
\end{rSection}

\begin{rSection}{Leadership $\&$ Extracurricular Activities}
\textbf{Student Union of School} \hfill{May 2015 - May 2016}\\
Minister of Publicity Department  \\
\textbf{Student Union of SJTU} \hfill{Nov. 2015 - Feb. 2016}\\
Minister of Communication Department of the Committee \\
\textbf{Outstanding Volunteers} \hfill{Shanghai International Marathon (2014$\&$2016), SJTU $120^{th}$ Anniversary}
\end{rSection}
\clearpage
%\end{CJK*}
\end{document}