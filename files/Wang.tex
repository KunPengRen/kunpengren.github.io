%%%%%%%%%%%%%%%%%%%%%%%%%%%%%%%%%%%%%%%%%
% Medium Length Professional CV
% LaTeX Template
% Version 2.0 (8/5/13)
%
% This template has been downloaded from:
% http://www.LaTeXTemplates.com
%
% Original author:
% Trey Hunner (http://www.treyhunner.com/)
%
% Important note:
% This template requires the resume.cls file to be in the same directory as the
% .tex file. The resume.cls file provides the resume style used for structuring the
% document.
%
%%%%%%%%%%%%%%%%%%%%%%%%%%%%%%%%%%%%%%%%%

%----------------------------------------------------------------------------------------
%	PACKAGES AND OTHER DOCUMENT CONFIGURATIONS
%----------------------------------------------------------------------------------------

\documentclass{resume} % Use the custom resume.cls style

\usepackage[left=0.75in,top=0.6in,right=0.75in,bottom=0.6in]{geometry} % Document margins
\usepackage{hyperref}
\usepackage{verbatim}
\usepackage{enumitem} 
\hypersetup{%
unicode=true, CJKbookmarks=true, bookmarksnumbered=true,
bookmarksopen=true, bookmarksopenlevel=1, breaklinks=true,
colorlinks=false, plainpages=false, pdfpagelabels, pdfborder=0 0 0 }
\name{Xuejian Wang} % Your name
\address{(+86)13262717148 \\ \href{http://xuejianwang.com}{xuejianwang.com} \\ \href{mailto:xjwang@apex.sjtu.edu.cn}{xjwang@apex.sjtu.edu.cn}}

\begin{document}
%\begin{CJK*}{GBK}{song}
\vspace{-1em}
%----------------------------------------------------------------------------------------
%	EDUCATION SECTION
%----------------------------------------------------------------------------------------
\begin{rSection}{RESEARCH INTERESTS}
The broad areas of machine learning and applications in various domains such as recommender system and network embedding. In general:
\begin{itemize}[itemindent=-5mm] 
	\itemsep=-5pt
	\vspace*{-5pt}
	\item[・] Mining underlying patterns of human interaction and behaviors
	\item[・] Developing practical machine learning algorithms and scalable tools to solve real-world problems.
\end{itemize}
\end{rSection}


\begin{rSection}{Education}
%{\bf }%\hfill { } \\
\textbf{Shanghai Jiao Tong University (SJTU), Shanghai, China} \hfill Sep. 2014 - Jun. 2018 (expected)\\ % \smallskip 
School of Electronic Information and Electric Engineering  \hfill GPA:3.45/4.0 \\
Bachelor of Science in Cyberspace Security \hfill (Junior GPA 3.56/4.0, Ranked 18/101)
\end{rSection}

%----------------------------------------------------------------------------------------
%	PUBLICATION SECTION
%----------------------------------------------------------------------------------------
\begin{rSection}{Publications}
%{\bf }%\hfill { } \\
$[1]$\textbf{Xuejian Wang}, Lantao Yu, Kan Ren, Guanyu Tao, Weinan Zhang, Yong Yu, and Jun Wang. Dynamic attention deep model for article recommendation by learning human editors' demonstration. In \emph{Proceedings of the 23rd ACM SIGKDD International Conference on Knowledge Discovery and Data Mining}, pages 2051-2059. \textbf{KDD 2017} \\
$[2]$Xuezhi Cao, Haokun Chen, \textbf{Xuejian Wang}, Weinan Zhang, Yong Yu. Neural Link Prediction over Aligned Networks, submitted to  \emph{Proc. AAAI}, 2018
\end{rSection}
%----------------------------------------------------------------------------------------
%	RESEARCH EXPERIENCE SECTION
%----------------------------------------------------------------------------------------
%\vspace{0.1em}
\begin{rSection}{Research Experience}
\begin{rSubsection}{Personalized Article Recommendation}{Sep. 2017 - Present}{Advisor: Prof. Yong Yu}{APEX Data $\&$ Knowledge Management Lab, SJTU}
\item This study focuses on recommending most relevant textual advertisements to users to maximize the Click-Through-Rate, in which the major challenge is how to match user interests with proper articles
\item Employing hierarchical attention mechanism to obtain article embeddings and construct user interests
\end{rSubsection}
\begin{rSubsection}{Neural Link Prediction over Aligned Networks}{Aug. 2017 - Sep. 2017}{Advisor: Prof. Yong Yu}{APEX Data $\&$ Knowledge Management Lab, SJTU}
\item This study explored link prediction over aligned networks, which are common in real world, yet little studied. The major challenge is the heterogeneousness of the considered networks, which could be addressed well by our proposed model.
\item Implemented baseline \emph{LINE: Large-scale Information Network Embedding} in Tensorflow and tuned parameters
\item Surveyed papers about social networks and contributed lots of submits to our final paper
\item Extended the idea of two aligned networks to multi-aligned networks which we left as future work
\end{rSubsection}
\begin{rSubsection}{Dynamic Attention Deep Model for Article Recommendation
by Learning Human Editors’ Demonstration}{Oct. 2016 - Feb. 2017}{Advisor: Prof. Weinan Zhang}{APEX Data $\&$ Knowledge Management Lab, SJTU}
\item This task is, essentially, a binary classification problem for articles, in which the major challenge is modeling the editors' underlying criterion varied with many factors such as time, current affairs, etc.
\item Employed attention mechanism to address data drift problem, resulting in more robust and stable predictions
\item Proposed a Dynamic Attention Deep Model (DADM) which outperformed other baselines in an A/B test
\item Our paper was accepted to KDD 2017 and the proposed DADM model was utilized in practical cases, automating the quality article selection process to alleviate the editors' working load
\end{rSubsection}
\end{rSection}
%----------------------------------------------------------------------------------------
%	INTERNSHIP
%----------------------------------------------------------------------------------------
\begin{rSection}{Professional Activities}
\begin{rSubsection}{Review Activities}{CIKM 2017, WSDM 2018, AAAI 2018}{}{}
\item Gave reviews for paper submissions in relevant domains for reference, which were mostly adopted
\item Learnt how to evaluate academic papers from a review perspective and developed tastes for papers
\end{rSubsection}
\end{rSection}


\begin{rSection}{Internship Experience}
\begin{rSubsection}{ULU Technologies Inc.}{Nov. 2016 - May 2017}{R$\&$D  Engineer Intern}{}
\item Developed a practical algorithm for article recommendation which is used in production
\item Improved coding ability, learned how to independently conduct experiments and developed communication skills
\item Got a main understanding of how AI start-up works
%\item Developed test cases and wrote general design documents for an industrial reliability testing platform.
%\item Programmed Oracle database in Java for a new testing module to update data and improve efficiency.
\end{rSubsection}
\end{rSection}


%----------------------------------------------------------------------------------------
%	TECHNICAL STRENGTHS SECTION
%----------------------------------------------------------------------------------------

\begin{rSection}{Awards}
\begin{rSubsection}{}{}{}{}
\item KDD Student Travel Award \hfill{2017}
\item Shanghai Jiao Tong University Excellent Scholarship \hfill{2017$\&$2016}
\item Excellent Student Award (Top 5\%)  \hfill{2016}
\item Second Prize, province-level CUMCM 2016 \hfill{2016}
\item Second Prize, ``Data Bang'' Data Innovation Contest (Top 2)\hfill{2015} 
\item Outstanding Undergraduate Practical Program (Top 5\%) \hfill{2015}
\end{rSubsection}
\end{rSection}

\begin{rSection}{Skills}
{\bf Machine Learning: }
\hspace*{3.0 cm} Tensorflow, Scikit-Learn, xGBoost, Keras\\
{\bf Programming Languages: }
\hspace*{1.8 cm} Python, MATLAB, C++, R, Verilog and \LaTeX  \\
%{\bf Standard Tests : }
%\hspace*{3.4 cm} TOEFL: 101 (R 29, L 26, S 20, W 26)  (To be updated)
\end{rSection}
\begin{rSection}{Leadership $\&$ Extracurricular Activities}
\textbf{SEIEE Student Union} \hfill{May 2015 - May 2016}\\
Minister of Publicity Department  \\
\textbf{SJTU Student Union} \hfill{Nov. 2015 - Feb. 2016}\\
Minister of Communication Department of the Committee \\
\textbf{Outstanding Volunteers} \hfill{Shanghai International Marathon (2014$\&$2016), SJTU $120^{th}$ Anniversary}
\end{rSection}
\clearpage
%\end{CJK*}
\end{document}